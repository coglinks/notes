% Options for packages loaded elsewhere
% Options for packages loaded elsewhere
\PassOptionsToPackage{unicode}{hyperref}
\PassOptionsToPackage{hyphens}{url}
%
\documentclass[
  ignorenonframetext,
]{beamer}
\newif\ifbibliography
\usepackage{pgfpages}
\setbeamertemplate{caption}[numbered]
\setbeamertemplate{caption label separator}{: }
\setbeamercolor{caption name}{fg=normal text.fg}
\beamertemplatenavigationsymbolsempty
% remove section numbering
\setbeamertemplate{part page}{
  \centering
  \begin{beamercolorbox}[sep=16pt,center]{part title}
    \usebeamerfont{part title}\insertpart\par
  \end{beamercolorbox}
}
\setbeamertemplate{section page}{
  \centering
  \begin{beamercolorbox}[sep=12pt,center]{section title}
    \usebeamerfont{section title}\insertsection\par
  \end{beamercolorbox}
}
\setbeamertemplate{subsection page}{
  \centering
  \begin{beamercolorbox}[sep=8pt,center]{subsection title}
    \usebeamerfont{subsection title}\insertsubsection\par
  \end{beamercolorbox}
}
% Prevent slide breaks in the middle of a paragraph
\widowpenalties 1 10000
\raggedbottom
\AtBeginPart{
  \frame{\partpage}
}
\AtBeginSection{
  \ifbibliography
  \else
    \frame{\sectionpage}
  \fi
}
\AtBeginSubsection{
  \frame{\subsectionpage}
}
\usepackage{iftex}
\ifPDFTeX
  \usepackage[T1]{fontenc}
  \usepackage[utf8]{inputenc}
  \usepackage{textcomp} % provide euro and other symbols
\else % if luatex or xetex
  \usepackage{unicode-math} % this also loads fontspec
  \defaultfontfeatures{Scale=MatchLowercase}
  \defaultfontfeatures[\rmfamily]{Ligatures=TeX,Scale=1}
\fi
\usepackage{lmodern}

\ifPDFTeX\else
  % xetex/luatex font selection
\fi
% Use upquote if available, for straight quotes in verbatim environments
\IfFileExists{upquote.sty}{\usepackage{upquote}}{}
\IfFileExists{microtype.sty}{% use microtype if available
  \usepackage[]{microtype}
  \UseMicrotypeSet[protrusion]{basicmath} % disable protrusion for tt fonts
}{}
\makeatletter
\@ifundefined{KOMAClassName}{% if non-KOMA class
  \IfFileExists{parskip.sty}{%
    \usepackage{parskip}
  }{% else
    \setlength{\parindent}{0pt}
    \setlength{\parskip}{6pt plus 2pt minus 1pt}}
}{% if KOMA class
  \KOMAoptions{parskip=half}}
\makeatother


\usepackage{longtable,booktabs,array}
\usepackage{calc} % for calculating minipage widths
\usepackage{caption}
% Make caption package work with longtable
\makeatletter
\def\fnum@table{\tablename~\thetable}
\makeatother
\usepackage{graphicx}
\makeatletter
\newsavebox\pandoc@box
\newcommand*\pandocbounded[1]{% scales image to fit in text height/width
  \sbox\pandoc@box{#1}%
  \Gscale@div\@tempa{\textheight}{\dimexpr\ht\pandoc@box+\dp\pandoc@box\relax}%
  \Gscale@div\@tempb{\linewidth}{\wd\pandoc@box}%
  \ifdim\@tempb\p@<\@tempa\p@\let\@tempa\@tempb\fi% select the smaller of both
  \ifdim\@tempa\p@<\p@\scalebox{\@tempa}{\usebox\pandoc@box}%
  \else\usebox{\pandoc@box}%
  \fi%
}
% Set default figure placement to htbp
\def\fps@figure{htbp}
\makeatother





\setlength{\emergencystretch}{3em} % prevent overfull lines

\providecommand{\tightlist}{%
  \setlength{\itemsep}{0pt}\setlength{\parskip}{0pt}}



 


\usepackage{makeidx}
\makeindex
\makeatletter
\@ifpackageloaded{caption}{}{\usepackage{caption}}
\AtBeginDocument{%
\ifdefined\contentsname
  \renewcommand*\contentsname{Table of contents}
\else
  \newcommand\contentsname{Table of contents}
\fi
\ifdefined\listfigurename
  \renewcommand*\listfigurename{List of Figures}
\else
  \newcommand\listfigurename{List of Figures}
\fi
\ifdefined\listtablename
  \renewcommand*\listtablename{List of Tables}
\else
  \newcommand\listtablename{List of Tables}
\fi
\ifdefined\figurename
  \renewcommand*\figurename{Figure}
\else
  \newcommand\figurename{Figure}
\fi
\ifdefined\tablename
  \renewcommand*\tablename{Table}
\else
  \newcommand\tablename{Table}
\fi
}
\@ifpackageloaded{float}{}{\usepackage{float}}
\floatstyle{ruled}
\@ifundefined{c@chapter}{\newfloat{codelisting}{h}{lop}}{\newfloat{codelisting}{h}{lop}[chapter]}
\floatname{codelisting}{Listing}
\newcommand*\listoflistings{\listof{codelisting}{List of Listings}}
\makeatother
\makeatletter
\makeatother
\makeatletter
\@ifpackageloaded{caption}{}{\usepackage{caption}}
\@ifpackageloaded{subcaption}{}{\usepackage{subcaption}}
\makeatother

\usepackage{bookmark}
\IfFileExists{xurl.sty}{\usepackage{xurl}}{} % add URL line breaks if available
\urlstyle{same}
\hypersetup{
  pdftitle={Presentation Title},
  hidelinks,
  pdfcreator={LaTeX via pandoc}}


\title{Presentation Title}
\author{}
\date{}

\begin{document}
\frame{\titlepage}


\begin{frame}{Topic 7: Network Layer}
\phantomsection\label{topic-7-network-layer}
\end{frame}

\begin{frame}{Introduction to Networks v7.0 (ITN)}
\phantomsection\label{introduction-to-networks-v7.0-itn}
\note{Cisco Networking Academy Program Introduction to Networks v7.0
(ITN) Module 8: Protocols and Modules}
\end{frame}

\begin{frame}{Topics}
\phantomsection\label{topics}
\begin{longtable}[]{@{}
  >{\centering\arraybackslash}p{(\linewidth - 2\tabcolsep) * \real{0.5000}}
  >{\centering\arraybackslash}p{(\linewidth - 2\tabcolsep) * \real{0.5000}}@{}}
\toprule\noalign{}
\begin{minipage}[b]{\linewidth}\centering
\textbf{Topic Title}
\end{minipage} & \begin{minipage}[b]{\linewidth}\centering
\textbf{Topic Objective}
\end{minipage} \\
\midrule\noalign{}
\endhead
\textbf{Network Layer Characteristics} & Explain how the network layer
uses IP protocols for reliable communications. \\
\textbf{IPv4 Packet} & Explain the role of the major header fields in
the IPv4 packet. \\
\textbf{IPv6 Packet} & Explain the role of the major header fields in
the IPv6 packet. \\
\textbf{How a Host Routes} & Explain how host devices use routing tables
to direct packets \\
\textbf{Introduction to Routing} & Explain how router use routing tables
to forward packets. \\
\textbf{Basic Router Configuration} & Configure initial settings,
interfaces and default gateway \\
\bottomrule\noalign{}
\end{longtable}

\note{8 -- Network Layer 8.0 -- Introduction 8.0.2 -- What will I learn
to do in this module?}
\end{frame}

\begin{frame}{Network Layer CharacteristicsThe Network Layer}
\phantomsection\label{network-layer-characteristics-the-network-layer}
\pandocbounded{\includegraphics[keepaspectratio]{t7-network-layer-img/t7-network-layer-v6_0.png}}

\begin{itemize}
\tightlist
\item
  Provides services to allow end devices to exchange data.
\item
  IP version 4 (IPv4) and IP version 6 (IPv6) are the principle network
  layer communication protocols.
\item
  {The network layer performs four basic operations:}

  \begin{itemize}
  \tightlist
  \item
    Addressing end devices
  \item
    Encapsulation
  \item
    Routing
  \item
    De-encapsulation ::: \{.notes\}
  \end{itemize}
\end{itemize}

8 -- Network Layer 8.1 -- Network Layer Characteristics 8.1.1 -- The
Network Layer

:::
\end{frame}

\begin{frame}{Network Layer CharacteristicsIP Encapsulation}
\phantomsection\label{network-layer-characteristics-ip-encapsulation}
\pandocbounded{\includegraphics[keepaspectratio]{t7-network-layer-img/t7-network-layer-v6_1.png}}

{IP encapsulates the transport layer segment.}

{IP packet will be examined by all layer 3 devices as it traverses the
network.}

{The IP addressing does not change from source to destination.}

\note{8 -- Network Layer 8.1 -- Network Layer Characteristics 8.1.2 --
IP Encapsulation}
\end{frame}

\begin{frame}{Network Layer CharacteristicsConnectionless}
\phantomsection\label{network-layer-characteristics-connectionless}
\emph{IP is Connectionless - There is no connection with the destination
established before sending data packets.}

Connectionless communication is conceptually {similar to sending a
letter } to someone without notifying the recipient in advance.

IP requires no initial exchange of control information to establish an
end-to-end connection before packets are forwarded.

\pandocbounded{\includegraphics[keepaspectratio]{t7-network-layer-img/t7-network-layer-v6_2.png}}

\pandocbounded{\includegraphics[keepaspectratio]{t7-network-layer-img/t7-network-layer-v6_3.png}}

\note{8 -- Network Layer 8.1 -- Network Layer Characteristics 8.1.4 --
Connectionless}
\end{frame}

\begin{frame}{Network Layer CharacteristicsBest Effort}
\phantomsection\label{network-layer-characteristics-best-effort}
\emph{IP is Best Effort - IP is inherently unreliable because packet
delivery is not guaranteed.}

\pandocbounded{\includegraphics[keepaspectratio]{t7-network-layer-img/t7-network-layer-v6_4.png}}

\begin{itemize}
\tightlist
\item
  With no pre-established end-to-end connection, senders are unaware
  whether destination devices:

  \begin{itemize}
  \tightlist
  \item
    Are present and functional when sending packets,
  \item
    Receives the packet
  \item
    Are able to access and read the packet.
  \end{itemize}
\item
  {The IP protocol was not designed to track and manage the flow of
  packets. These functions, if required, are performed by other
  protocols at other layers, primarily TCP at Layer 4.} ::: \{.notes\}
\end{itemize}

8 -- Network Layer 8.1 -- Network Layer Characteristics 8.1.5 -- Best
Effort

:::
\end{frame}

\begin{frame}{Network Layer CharacteristicsMedia Independent}
\phantomsection\label{network-layer-characteristics-media-independent}
\begin{itemize}
\tightlist
\item
  \emph{IP is media Independent: Operation is independent of the medium
  (i.e., copper, fiber-optic, or wireless) carrying the data.}
\end{itemize}

\pandocbounded{\includegraphics[keepaspectratio]{t7-network-layer-img/t7-network-layer-v6_5.png}}

\begin{itemize}
\tightlist
\item
  IP does not concern itself with the type of frame required at the data
  link layer or the media type at the physical layer.
\item
  {As shown in the figure, IP packets can be sent over any media type:
  copper, fiber, or wireless.} ::: \{.notes\}
\end{itemize}

8 -- Network Layer 8.1 -- Network Layer Characteristics 8.1.6 -- Media
Independent 8.1.7 Check Your Understanding -- IP Characteristics

:::
\end{frame}

\begin{frame}{IPv4 PacketIPv4 Packet Header}
\phantomsection\label{ipv4-packet-ipv4-packet-header}
Diagram is read from left to right, 4 bytes per line

\emph{The two most commonly referenced fields are the source and
destination IP addresses} .

These fields identify where the packet is coming from and where it is
going.

Typically, \emph{these addresses do not change while travelling from the
source to the destination.}

\pandocbounded{\includegraphics[keepaspectratio]{t7-network-layer-img/t7-network-layer-v6_6.png}}

\note{8 -- Network Layer 8.2 -- IPv4 Packet 8.2.1 -- IPv4 Packet Header}
\end{frame}

\begin{frame}{IPv4 PacketIPv4 Packet Header Fields}
\phantomsection\label{ipv4-packet-ipv4-packet-header-fields}
Significant fields in the IPv4 header:

\begin{longtable}[]{@{}
  >{\centering\arraybackslash}p{(\linewidth - 2\tabcolsep) * \real{0.5000}}
  >{\centering\arraybackslash}p{(\linewidth - 2\tabcolsep) * \real{0.5000}}@{}}
\toprule\noalign{}
\begin{minipage}[b]{\linewidth}\centering
Function
\end{minipage} & \begin{minipage}[b]{\linewidth}\centering
Description
\end{minipage} \\
\midrule\noalign{}
\endhead
\textbf{Version} & This will be for v4, as opposed to v6, a 4 bit field=
0100 \\
\textbf{Differentiated Services} & {Used for } {QoS} : DiffServ -- DS
field or the older IntServ -- ToS or Type of Service \\
\textbf{Header Checksum} & Detect corruption in the IPv4 header \\
\textbf{Time to Live (TTL)} & {Layer 3 hop count. When it becomes zero
the router} { will discard the packet.} \\
\textbf{Protocol} & I.D.s next level protocol: ICMP, TCP, UDP, etc. \\
{ \textbf{Source IPv4 Address} } & 32 bit source address \\
{ \textbf{Destination IPv4 Address} } & 32 bit destination address \\
\bottomrule\noalign{}
\end{longtable}

\note{8 -- Network Layer 8.2 -- IPv4 Packet 8.2.2 -- IPv4 Packet Header
Fields}
\end{frame}

\begin{frame}{IPv6 PacketsLimitations of IPv4}
\phantomsection\label{ipv6-packets-limitations-of-ipv4}
\begin{itemize}
\tightlist
\item
  {IPv4 address depletion } -- We have basically run out of IPv4
  addressing.
\item
  {Lack of end-to-end connectivity } -- To make IPv4 survive this long,
  private addressing and NAT were created. This ended direct
  communications with public addressing.
\item
  {Increased network complexity } -- NAT was meant as temporary solution
  and creates issues on the network as a side effect of manipulating the
  network headers addressing. {NAT causes latency and troubleshooting
  issues} . ::: \{.notes\}
\end{itemize}

8 -- Network Layer 8.3 -- IPv6 Packets 8.3.1 -- Limitations of IPv4

:::
\end{frame}

\begin{frame}{IPv6 PacketsIPv6 Overview}
\phantomsection\label{ipv6-packets-ipv6-overview}
\pandocbounded{\includegraphics[keepaspectratio]{t7-network-layer-img/t7-network-layer-v6_7.png}}

\begin{itemize}
\tightlist
\item
  Developed to overcomes the limitations of IPv4.
\item
  {IPv6 improvements:}

  \begin{itemize}
  \tightlist
  \item
    {Increased address space } -- based on 128 bit address, not 32 bits
  \item
    {Improved packet handling } -- simplified header with fewer fields
  \item
    {Eliminates the need for NAT } -- since there is a huge amount of
    addressing, there is no need to use private addressing internally
    and be mapped to a shared public address
  \end{itemize}
\end{itemize}

\pandocbounded{\includegraphics[keepaspectratio]{t7-network-layer-img/t7-network-layer-v6_8.png}}

\note{8 -- Network Layer 8.3 -- IPv6 Packets 8.3.2 -- IPv6 Overview}
\end{frame}

\begin{frame}{IPv6 PacketsFields in the IPv6 Packet Header}
\phantomsection\label{ipv6-packets-fields-in-the-ipv6-packet-header}
\pandocbounded{\includegraphics[keepaspectratio]{t7-network-layer-img/t7-network-layer-v6_9.png}}

\begin{itemize}
\tightlist
\item
  The IPv6 header is simplified, but not smaller.
\item
  The header is fixed {at 40 Bytes} long.
\item
  Some IPv4 fields were removed to improve performance:

  \begin{itemize}
  \tightlist
  \item
    Flag
  \item
    Fragment Offset
  \item
    Header Checksum ::: \{.notes\}
  \end{itemize}
\end{itemize}

8 -- Network Layer 8.3 -- IPv6 Packets 8.3.3 -- IPv4 Packet Header
Fields in the IPv6 Packet Header

:::
\end{frame}

\begin{frame}{IPv6 PacketsIPv6 Packet Header}
\phantomsection\label{ipv6-packets-ipv6-packet-header}
\begin{itemize}
\tightlist
\item
  Significant fields in the IPv6 header:
\end{itemize}

\begin{longtable}[]{@{}
  >{\centering\arraybackslash}p{(\linewidth - 2\tabcolsep) * \real{0.5000}}
  >{\centering\arraybackslash}p{(\linewidth - 2\tabcolsep) * \real{0.5000}}@{}}
\toprule\noalign{}
\begin{minipage}[b]{\linewidth}\centering
Function
\end{minipage} & \begin{minipage}[b]{\linewidth}\centering
Description
\end{minipage} \\
\midrule\noalign{}
\endhead
\textbf{Version} & This will be for v6, as opposed to v4, a 4 bit field=
0110 \\
\textbf{Traffic Class} & Used for QoS: Equivalent to DiffServ -- DS
field \\
\textbf{Flow Label} & Informs device to handle identical flow labels the
same way, 20 bit field \\
\textbf{Payload Length} & This 16-bit field indicates the length of the
data portion or payload of the IPv6 packet \\
\textbf{Next Header} & I.D.s next level protocol: ICMP, TCP, UDP,
etc. \\
\textbf{Hop Limit} & Replaces TTL field Layer 3 hop count \\
\textbf{Source IPv6 Address} & 128 bit source address \\
\textbf{Destination IPV6 Address} & 128 bit destination address \\
\bottomrule\noalign{}
\end{longtable}

\note{8 -- Network Layer 8.3 -- IPv6 Packets 8.3.4 -- IPv6 Packet
Header}
\end{frame}

\begin{frame}{How a Host RoutesHost Forwarding Decision}
\phantomsection\label{how-a-host-routes-host-forwarding-decision}
\begin{itemize}
\tightlist
\item
  Each host device creates its own routing table.
\item
  A host can send packets to the following destinations:

  \begin{itemize}
  \tightlist
  \item
    Itself -- {127.0.0.1} (IPv4) -- the special address of the {loopback
    interface}
  \item
    Local Hosts -- on the same LAN
  \item
    Remote Hosts -- on the different LANs
  \end{itemize}
\item
  {The source device determines whether the destination is local or
  remote by comparing its own IP address with the destination IP
  address.}
\item
  {Remote traffic is forwarded directly to the } { \emph{default gateway
  } } {on the LAN.}
\end{itemize}

\pandocbounded{\includegraphics[keepaspectratio]{t7-network-layer-img/t7-network-layer-v6_10.png}}

Default Gateway

{Intermediate Device}

\note{8 -- Network Layer 8.4 -- How a Host Routes 8.4.1 -- Host
Forwarding Decision (Cont.)}
\end{frame}

\begin{frame}{How a Host RoutesDefault Gateway}
\phantomsection\label{how-a-host-routes-default-gateway}
\begin{itemize}
\tightlist
\item
  A router or layer 3 switch can be a default-gateway.
\item
  {Features of a default gateway (DGW)} :

  \begin{itemize}
  \tightlist
  \item
    It must have an IP address in the same range as the rest of the LAN.
  \item
    It can accept data from the LAN and forward traffic to other
    networks.
  \end{itemize}
\item
  {If a device has no default gateway or a bad default gateway, its
  traffic will not be able to leave the LAN.}
\end{itemize}

\pandocbounded{\includegraphics[keepaspectratio]{t7-network-layer-img/t7-network-layer-v6_11.png}}

\textbf{Local Network}

192.168.10.0/24

The host will know the default gateway either statically or through DHCP
in IPv4.

All devices on the LAN use the DGW to send traffic remotely.

\note{8 -- Network Layer 8.4 -- How a Host Routes 8.4.2 -- Default
Gateway}
\end{frame}

\begin{frame}{How a Host Routes Host Routing Tables}
\phantomsection\label{how-a-host-routes-host-routing-tables}
\pandocbounded{\includegraphics[keepaspectratio]{t7-network-layer-img/t7-network-layer-v6_12.png}}

\begin{itemize}
\tightlist
\item
  Use {route print } or {netstat -r} to display the routing table.
\item
  Three sections:

  \begin{itemize}
  \tightlist
  \item
    Interface List and MAC addressing
  \item
    IPv4 Routing Table
  \item
    IPv6 Routing Table
  \end{itemize}
\end{itemize}

IPv4 Routing Table for PC1

\pandocbounded{\includegraphics[keepaspectratio]{t7-network-layer-img/t7-network-layer-v6_13.png}}

{Used to determine which route is used if there are more than one route
to the same destination.}

Default gateway for PC1

\note{8 -- Network Layer 8.4 -- How a Host Routes 8.4.4 -- Host Routing
Tables 8.4.5 -- Check Your Understanding -- How a Host Routes}
\end{frame}

\begin{frame}{Introduction to RoutingRouter Packet Forwarding Decision}
\phantomsection\label{introduction-to-routing-router-packet-forwarding-decision}
\pandocbounded{\includegraphics[keepaspectratio]{t7-network-layer-img/t7-network-layer-v6_14.png}}

\begin{itemize}
\tightlist
\item
  \emph{What happens when the router receives the frame from the host
  device?}
\end{itemize}

\textbf{R1 Routing Table}

\pandocbounded{\includegraphics[keepaspectratio]{t7-network-layer-img/t7-network-layer-v6_15.png}}

{ \emph{Packet arrives on the Gigabit Ethernet 0/0/0 interface of router
R1. R1 de-encapsulates the Layer 2 Ethernet header and trailer.} }

{ \emph{Router R1 examines the destination IPv4 address of the packet
and searches for the best match in its IPv4 routing table. The route
entry indicates that this packet is to be forwarded to router R2.} }

{ \emph{Router R1 encapsulates the packet into a } } \emph{new Ethernet
header and trailer} { \emph{, and forwards the packet to the next hop
router R2.} }

\note{8 -- Network Layer 8.5 -- Introduction to Routing 8.5.1 -- Router
Packet Forwarding Decision}
\end{frame}

\begin{frame}{Introduction to RoutingIP Router Routing Table}
\phantomsection\label{introduction-to-routing-ip-router-routing-table}
\begin{itemize}
\tightlist
\item
  There three {types of routes } in a routing table:
\item
  \textbf{Directly Connected networks } -- {Are automatically added by
  the router, provided the interface is active and has IP addressing.}
\item
  \textbf{Remote networks} -- Routes that the router does not have a
  direct connection and may be {learned:}

  \begin{itemize}
  \tightlist
  \item
    {Manually -- } {with a static route}
  \item
    {Dynamically } -- {by using a dynamic routing protocol}
  \end{itemize}
\item
  \textbf{Default Route } -- {a gateway of last resort} , used when
  there is not a match in the routing table
\end{itemize}

\pandocbounded{\includegraphics[keepaspectratio]{t7-network-layer-img/t7-network-layer-v6_16.png}}

\note{8 -- Network Layer 8.5 -- Introduction to Routing 8.5.2 -- IP
Router Routing Table}
\end{frame}

\begin{frame}{Introduction to RoutingStatic Routing}
\phantomsection\label{introduction-to-routing-static-routing}
\pandocbounded{\includegraphics[keepaspectratio]{t7-network-layer-img/t7-network-layer-v6_17.png}}

Static Route Characteristics:

Must be {configured manually}

Must be {adjusted manually} by the administrator {when there is a change
in the topology}

{Good for small non-redundant networks}

Often used in conjunction with a dynamic routing protocol for
configuring a default route

{R1 is configured with a static route to reach the 10.1.1.0l24 network.
If this path changes, R1 will require a new static route.}

\pandocbounded{\includegraphics[keepaspectratio]{t7-network-layer-img/t7-network-layer-v6_18.png}}

{If the route from R1 via R2 is no longer available, a new static route
via R3 would need to be configured.}

\note{8 -- Network Layer 8.5 -- Introduction to Routing 8.5.3 -- Static
Routing}
\end{frame}

\begin{frame}{Introduction to RoutingDynamic Routing}
\phantomsection\label{introduction-to-routing-dynamic-routing}
\pandocbounded{\includegraphics[keepaspectratio]{t7-network-layer-img/t7-network-layer-v6_19.png}}

Dynamic Routes Automatically:

Discover remote networks

Maintain up-to-date information

Choose the best path to the destination

Find new best paths when there is a topology change

R1 is using the OSPF routing protocol to let R2 know about the
192.168.10.0/24 network.

R2 is using the OSPF routing protocol to let R1 know about the
10.1.1.0/24 network.

\pandocbounded{\includegraphics[keepaspectratio]{t7-network-layer-img/t7-network-layer-v6_20.png}}

{R1, R2, and R3 are using the OSPF dynamic routing protocol. }

{If there is network topology change, they can automatically adjust to
find a new path.}

\note{8 -- Network Layer 8.5 -- Introduction to Routing 8.5.4 -- Dynamic
Routing}
\end{frame}

\begin{frame}{Introduction to RoutingIntroduction to an IPv4 Routing
Table}
\phantomsection\label{introduction-to-routing-introduction-to-an-ipv4-routing-table}
\pandocbounded{\includegraphics[keepaspectratio]{t7-network-layer-img/t7-network-layer-v6_21.png}}

\pandocbounded{\includegraphics[keepaspectratio]{t7-network-layer-img/t7-network-layer-v6_22.png}}

\begin{itemize}
\tightlist
\item
  The { \emph{show ip route } } command displays:
\item
  {Route sources:}

  \begin{itemize}
  \tightlist
  \item
    {L} - Directly connected local interface IP address
  \item
    {C} -- Directly connected network
  \item
    {S} -- Static route, manually configured
  \item
    {O} -- OSPF
  \item
    {D} -- EIGRP
  \end{itemize}
\item
  {Types of routes:}

  \begin{itemize}
  \tightlist
  \item
    Directly Connected -- {C} and {L}
  \item
    {Remote Routes } -- {O} , {D} , etc.
  \item
    Default Routes -- {S*} ::: \{.notes\}
  \end{itemize}
\end{itemize}

8 -- Network Layer 8.5 -- Introduction to Routing 8.5.6 -- Introduction
to an IPv4 Routing Table 8.5.7 -- Check Your Understanding --
Introduction to Routing

:::
\end{frame}

\begin{frame}{Configure Initial Router SettingsBasic Router
Configuration Example}
\phantomsection\label{configure-initial-router-settings-basic-router-configuration-example}
R1(config)\# \textbf{hostname R1}

R1(config)\# \textbf{enable secret class}

R1(config)\# \textbf{line console 0}

R1(config-line)\# \textbf{password cisco}

R1(config-line)\# \textbf{login}

R1 \textbf{(} config-line)\# \textbf{line vty 0 4}

R1(config-line)\# \textbf{password cisco}

R1(config-line)\# \textbf{login}

R1(config-line)\# \textbf{transport input ssh telnet}

R1(config-line)\# \textbf{exit}

R1(config)\# \textbf{service password encryption}

R1(config)\# \textbf{banner motd \#}

Enter TEXT message. End with a new line and the \#

**********************************************

WARNING: Unauthorized access is prohibited!

**********************************************

R1\# \textbf{copy running-config startup-config}

{Commands for basic router configuration}

{Configuration is saved to NVRAM.}

\note{10 -- Basic Router Configuration 10.1 -- Configure Initial Router
Settings 10.1.2 -- Basic Routing Configuration Example 10.1.3 - Syntax
Checker -- Configure Initial Router Settings}
\end{frame}

\begin{frame}{Configure InterfacesConfigure Router Interfaces Example}
\phantomsection\label{configure-interfaces-configure-router-interfaces-example}
\pandocbounded{\includegraphics[keepaspectratio]{t7-network-layer-img/t7-network-layer-v6_23.png}}

{ \textbf{The commands to configure interface G0/0/0 on R1:} }

R1(config)\# \textbf{interface gigabitEthernet 0/0/0}

R1(config-if)\# \textbf{description Link to LAN}

R1(config-if)\# { \textbf{ip} } \_\_ address 192.168.10.1
255.255.255.0\_\_

R1(config-if)\# { \textbf{ipv6} } \_\_ address
2001:db8:acad:10::1/64\_\_

R1(config-if)\# { \textbf{no shutdown } }

R1(config-if)\# \textbf{exit}

R1(config)\#

*Aug 1 01:43:53.435: \%LINK-3-UPDOWN: Interface GigabitEthernet0/0/0,
changed state to down

*Aug 1 01:43:56.447: \%LINK-3-UPDOWN: Interface GigabitEthernet0/0/0,
changed state to up

*Aug 1 01:43:57.447: \%LINEPROTO-5-UPDOWN: Line protocol on Interface
GigabitEthernet0/0/0, changed state to up

adds information about the network connected to the interface

{activates the interface}

\note{10 -- Basic Router Configuration 10.2 -- Configure Interfaces
10.2.2 -- Configure Router Interfaces Example}
\end{frame}

\begin{frame}{Configure InterfacesVerify Interface Configuration}
\phantomsection\label{configure-interfaces-verify-interface-configuration}
{To verify interface configuration or to view status of all interfaces
use the } { \textbf{show ip interface brief } } {and } { \textbf{show
ipv6 interface brief } } {commands shown here:}

{R1\# } { \textbf{show ip interface brief} }

Interface IP-Address OK? Method Status Protocol

GigabitEthernet0/0/0 192.168.10.1 YES manual up up

GigabitEthernet0/0/1 209.165.200.225 YES manual up up

Vlan1 unassigned YES unset administratively down down

{R1\# } { \textbf{show ipv6 interface brief} }

GigabitEthernet0/0/0 {[}up/up{]}

FE80::201:C9FF:FE89:4501

2001:DB8:ACAD:10::1

GigabitEthernet0/0/1 {[}up/up{]}

FE80::201:C9FF:FE89:4502

2001:DB8:FEED:224::1

Vlan1 {[}administratively down/down{]}

unassigned

\note{10 -- Basic Router Configuration 10.2 -- Configure Interfaces
10.2.3 -- Verify Interface Configuration}
\end{frame}

\begin{frame}{Configure the Default GatewayDefault Gateway on a Host}
\phantomsection\label{configure-the-default-gateway-default-gateway-on-a-host}
\pandocbounded{\includegraphics[keepaspectratio]{t7-network-layer-img/t7-network-layer-v6_24.png}}

{The default gateway is used when a host sends a packet to a device on
another network.}

{The default gateway address is generally the router interface address
attached to the local network of the host.}

{To reach PC3, PC1 addresses a packet with the IPv4 address of PC3, but
forwards the packet to its default gateway, the G0/0/0 interface of R1.}

{Network address of LAN1}

{ \textbf{Note} } {: The IP address of the host and the router interface
must be in the same network.}

\note{10 -- Basic Router Configuration 10.3 -- Configure the Default
Gateway 10.3.1 -- Default Gateway on a Host}

\pandocbounded{\includegraphics[keepaspectratio]{t7-network-layer-img/t7-network-layer-v6_25.png}}

{A switch must have a default gateway address configured to } {remotely
} {manage the switch } {from another network} {. }

Configured by using the { \textbf{ip} } { \_\_ default-gateway \_\_ }
command.

\note{10 -- Basic Router Configuration 10.3 -- Configure the Default
Gateway 10.3.2 -- Default Gateway on a Switch 10.3.3 -- Syntax Checker
-- Configure the Default Gateway}
\end{frame}



\printindex


\end{document}
