\documentclass[openany]{book}

\usepackage{inctech}

\begin{document}

% Front matter
\tableofcontents

% Main content
\mainmatter

\raggedright

\chapter{To Study}

Equalities

Logarithms

Trigonometry

Convert a function to piecewise functions \& vice-versa


\part{Number Theory}

\chapter{null}

\section{Types}

\begin{outline}
	\item {\bf Real numbers}, $\mathbb{R}\supset \{ \mathbb{Q,\ (R-Q)} \} $
	
	\begin{outline}

		\item Natural numbers,
		$ \mathbb{N}=\{\ x : 1\leq x \leq\infty\ \} $
		\item Whole numbers,
		$ \mathbb{W}=\{\ x : 0\leq x \leq\infty\ \} $
		\item Integers,
		$ \mathbb{I}=\{\ x: -\infty\leq x \leq\infty\ \} $
		\item Rational numbers,
		$ \mathbb{Q}=\{\ x : x = \dfrac{p}{q}, where\ p,q \in I\ \&\ q \neq 0\ \} $
		\item Irrational numbers,
		$ \mathbb{(R-Q)}=\{\ x : x \neq \dfrac{p}{q}, where\ p,q \in I\ \&\ q \neq 0\ \} $
		
		Note that irrational numbers are non-terminating and non-repeating decimals.
		
	\end{outline}
	
	\item {\bf Imaginary numbers}, \textit{Im}
	\( =\{ x : x = ai\ where\ a \in \mathbb{R}\ \&\ i=\sqrt{-1}\ \} \)
	
	\begin{outline}

		\item Complex Numbers,
		\( \mathbb{C}=\{ x : x=(a+bi),\ where\
		a,b \in \mathbb{R}\ \&\ i=\sqrt{-1}\ \} \)
		
	\end{outline}
	
\end{outline}

\chapter{Prime Numbers}

\section{To find a prime number}

Via AKS test

\begin{theorem}
	To find a composite number (x) it's prime factors must be less than the square root of the number.
\end{theorem}

\subsection{Some tips}

\begin{itemize}

	\item All primes are odd numbers, except 2.
	\item All even numbers are not primes. No primes end in zero, nor 5.\\
	      This bit of trivia can help you to identify non-primes, at least.
	\item So all primes end in 1, 3, 7, or 9 (the reverse is not true though).
	      
\end{itemize}

\chapter{Addition}

\section{LCD}

To find LCD:\@
\begin{enumerate}
	\item Find the LCM of the denominators
	\item Multiply the (numerator and the denominator) of each dfraction with a number till denominator becomes LCM.\@
\end{enumerate}

\href{https://www.youtube.com/watch?v=Ws03IbNrjfM}{To find LCM of polynomials}

\part{Geometry}

\chapter{Graphs}

Determine the intervals that a graph is increasing and decreasing

ref:
https://www.youtube.com/watch?v=d3mDMFWcBo8

\chapter{Angles}

\section{Types}

Single angle types:
\begin{itemize}
	\item Acute Angle: \( \theta < 90 \)
	\item Right Angle: \( \theta = 90 \)
	\item Obtuse Angle: \( 90 < \theta < 180 \)
	\item Straight Angle: \( \theta = 180 \)
	\item Reflex Angle: \( 180 < \theta < 360 \)
	\item Full Angle:  \( \theta = 360 \)
	      
\end{itemize}

Combinations:
\begin{itemize}
	\item Supplementary angles: 2 angles add to 180
	\item Complimentary angles: 2 angles add to 90
	\item \href{https://www.cuemath.com/geometry/coterminal-angles/}{Co-terminal angles}: Angles that have the same initial side and share the terminal sides.
	\item Reference angle: the smallest possible angle made by the terminal side of the given angle with the x-axis.
	      
	      ref:
	      
	      \href{https://www.piday.org/calculators/reference-angle-calculator/}{Article with everything}
	      
	      \href{https://www.cuemath.com/geometry/reference-angle/}{Scroll for trick to find ref. angle}
	      
	      \url{https://www.youtube.com/watch?v=rhLN7doUCAQ}
	      
	      \url{https://www.youtube.com/watch?v=E-xFXpVo14o}
	      
	      \url{https://www.youtube.com/watch?v=qQLeITxH3z8}
	      
\end{itemize}


\chapter{Area}

\section{3D}
\begin{table}
	\centering
	\begin{tabular}{cccc}
		\toprule
		Shape      & TSA               & LSA/CSA         & Volume                       \\
		\midrule
		Cuboid     & \(2( lb+bh+hl)\)  & \(2h (l + b)\)  & \( lbh \)                    \\
		Cube       & \(6a^2\)          & \(4a^2\)        & \(a^3\)                      \\
		Cylinder   & \( 2\pi r(r+h) \) & \(2\pi rh\)     & \( \pi r^2 h \)              \\
		Sphere     & \( 4\pi r^2 \)    & -               & \( \dfrac{4}{3}\pi r^3 \)    \\
		Hemisphere & \( 3 \pi r^2 \)   & \( 2 \pi r^2 \) & \( \dfrac{2}{3}\pi r^3 \)    \\
		Cone       & \( \pi r (r+l) \) & \( \pi rl \)    & \( \dfrac{1}{3} \pi r^2 h \) \\
		\bottomrule
	\end{tabular}
	\label{tab:3D}
\end{table}

\chapter{Straight Lines}

Standard representations:

\begin{multline*}
	y=mx+c\\
	Ax+By=c\\
	y-y_1=m(x-x_1)\\
	\dfrac{x}{a}+\dfrac{y}{b}=1
\end{multline*}

\begin{theorem}
	The slopes of two perpendicular lines are negative reciprocals of each other
\end{theorem}

Mid-point formula:
\[
	(x_1, y_1), (x_2, y_2) \implies \left[\  \left( \dfrac{x_2 + x_1}{2} \right), \left( \dfrac{y_2 + y_1}{2} \right)\ \right]
\]

Distance formula:
\[
	d=\sqrt{ (x_2 - x_1)^2 + (y_2 - y_1)^2 }
\]

\chapter{Conic Sections}

Types:
\begin{itemize}
	\item Circles
	\item Parabola
	\item Ellipses
	\item Hyperbolas
\end{itemize}

\section{Circles}

Definition:-

The set of all points equidistant from a point

Let:
\begin{itemize}
	\item Centre of a circle \(=(h.k)\)
	\item \(\mathrm{r}=\) radius
	\item \((x,y)=\) an arbitrary point on the circle
\end{itemize}
Then (by distance formula):
\begin{flalign*}
	 & r=\sqrt{(x-h)^2+(y-k)^2} \\
	 & r^2=(x-h)^2+(y-k)^2      \\
\end{flalign*}

Standard equation of a circle: 
\[r^2=(x-h)^2+(y-k)^2,\ where\ r>0\]

\section{Parabola}

Definitions:-

Parabola: The set of all points equidistant from a fixed line (directrix) \& a fixed point (focus) not on the directrix

Vertex: the highest or lowest point of a parabola

Axis of symmetry: line that passes through the vertex \& is perpendicular to the directrix.

Focal length (\(|p|\)): is the distance b/w focus and vertex.

\begin{tabular}{ccL}
	\toprule
	                   & Type                                    & \text{Formula}      \\
	\midrule
	Quadratic Equation & standard form                           & y=ax^2+bx+c         \\
	\midrule
	Parabola           & standard form                           & y=ax^2+bx+c         \\
	                   & vertex form (vertical-axis symmetry)    & y=a(x-h)^2+k        \\
	                   & vertex form (hoorizontal-axis symmetry) & x=a(y-k)^2+h        \\
	                   & Factored or Intercept form              & y = a(x - p)(x - q) \\
	\bottomerule
\end{tabular}

\begin{tabular}{lLL}
	\hline
	\multicolumn{3}{|c|}{Standard form of an Equation of parabola with vertex at the \((h,\ k)\)} \\
	\hline
	                 & \text{Vertical parabola} & \text{Horizontal parabola}                      \\
	\hline
	Equation         & 
	x^2=4py          & 
	y^2=4px                                                                                       \\
	Vertex           & 
	(h,k)            & 
	(h,k)                                                                                         \\
	Focus            & 
	(h,k+p)          & 
	(h+p,k)                                                                                       \\
	Directrix        & 
	y=k-p            & 
	x=h-p                                                                                         \\
	Axis of symmetry & 
	x=h              & 
	y=k                                                                                           \\
	Graph: \((p>0)\)                                                                              \\
	Graph: \((p<0)\)                                                                              \\
	\hline
\end{tabular}

\subsection{For vertical symmetry}

In Intercept form, p \& q are both x-intercepts

For a Parabola:
\begin{enumerate}
	\item In standard form:
	      \begin{enumerate}
		      \item Vertex, \((x,y)=(\dfrac{-b}{2a},\ f(\dfrac{-b}{2a}))\)
	      \end{enumerate}
	\item in vertex form:
	      \begin{enumerate}
		      \item Axis of Symmetry, \(x=h\)
		      \item Vertex, \((x,y)=(h, k)\)
	      \end{enumerate}
\end{enumerate}

\subsection{For Horizontal Symmetry}

For a Parabola:
If:
\begin{itemize}
	\item \(a>0\), the parabola opens to the right
	\item \(a<0\), the parabola opens to the left
\end{itemize}
\begin{enumerate}
	\item In standard form:
	      \begin{enumerate} 
		      \item Vertex, \((x,y)=(\dfrac{-b}{2a},\ f(\dfrac{-b}{2a}))\)
	      \end{enumerate}
	\item in vertex form:
	      \begin{enumerate}
		      \item Axis of Symmetry, \(y=k\)
		      \item Vertex, \((x,y)=(h, k)\)
	      \end{enumerate}
\end{enumerate}

\subsection{Vertex Formula}

\begin{itemize}
	\item \(y=ax^2+bx+c\)
	      \begin{itemize} 
		      \item x-coordinate of vertex, \(x=\dfrac{-b}{2a}\)
		      \item y-coordinate of vertex,  \(y=ax^2+bx+c\)
	      \end{itemize}
	\item \(x=ay^2+by+c\)
	      \begin{itemize}
		      \item y-coordinate of vertex, \(y=\dfrac{-b}{2a}\)
		      \item x-coordinate of vertex, \(x=ay^2+by+c\)
	      \end{itemize}
\end{itemize}

\section{Ellipse}

Def:

Ellipse: The set of all points \((x,y)\), such that the sum of the distances \((x,y)\) \& 2 distinct points (foci, sin., focus) is a constant.

\begin{tabular}{|c|L|}
	\hline
	Standard Equation   & \dfrac{x^2}{a^2}+\dfrac{y^2}{b^2}=1         \\
	\hline
	Ellipse not centered in origin                                    \\
	Form of equation    & \dfrac{(x+m)^2}{a^2}+\dfrac{(y+n)^2}{b^2}=1 \\ 
	\((x,y)\) of centre & (x=-m,y=-n)                                 \\
	x-intercept         & x=\pm a                                     \\
	y-intercept         & y=\pm b                                     \\
	\hline
\end{tabular}

\section{Hyperbola}

Def:

Hyperbola: The set of all points \((x,y)\) such that the absolute value of the difference of the b/w \((x,y)\) \& 2 distict points is a constant.

Transverse axis: The axis of symmetry of a hyperbola whose 2 vertices lie on.

\begin{tabular}{|c|L|}
	\hline
	Standard forms                                                                                         \\
	\hline
	Horizontal Transverse axis (x is +, branches open left \& right) & \dfrac{x^2}{a^2}-\dfrac{y^2}{b^2}=1 \\
	Vertical Transverse axis (y is +, branches open up \& down)      & \dfrac{y^2}{b^2}-\dfrac{x^2}{a^2}=1 \\
	\hline
\end{tabular}

\subsubsection{Graphing}

Done by drawing asymptotes to the diagonals of a reference rectangle built from a and b in standard form. The 4 corners are:

\[(a,b),\ (-a,b),\ (a,-b),\ (-a,-b)\]

\section{General Form of Conic Sections}



\part{Algebra}

\chapter{Mathematical Induction}

Principle of Mathematical Induction

Let \(P_n,\ where\ n \in ^+I \) and let \(k \in ^+I\). Then \(P_n\) is true for all n if:
\begin{enumerate}
	\item \(P_1\) is true
	\item The truth of \(P_k \Rightarrow \text{the truth of } P_{k+1}\)
\end{enumerate}

Extended Principle of Mathematical Induction

Let \(P_n,\ where\ n \in ^+I \) and let \(k \in ^+I\). Then \(P_n\) is true for all \(n \geq j\) if:
\begin{enumerate}
	\item \(P_j\) is true
	\item For \(k \in I,\ where\ k \geq j \), the truth of \(P_k \Rightarrow \text{the truth of } P_{k+1}\)
\end{enumerate}

\chapter{Sets}

\section{Some definitions}
\begin{description}
	\item Domain
	\item Range
	\item Co-domain
\end{description}

The number (b – a) is called the length of any of the intervals (a, b), [a, b], [a, b) or (a, b].\\
\(n[P(A)] = 2^m, m = n(A)\)\\
Some Properties of Complement Sets:\\
1. Complement laws: (i) \(A \cup A\text{'} = U \) (ii) \(A \cap A\text{'} = \phi\)
2. De Morgan’s law: (i) \((A \cup B)\text{'} = A\text{'} \cap B\text{'}\)(ii) $$(A \cap B)\text{'} = A\text{'} \cup B\text{'} $$
3. Law of double complementation : (A$\text{'}$)$\text{'}$ = A\\
4. Laws of empty set and universal set \(\phi\text{'}\) = U and U$\text{'}$ = $\phi$.\\

\section{\nameref{sec:DRoF}}

\chapter{Relations}

If n(A) = p and n(B) = q, then n(A × B) = pq\\
If (a, b) = (x, y), then a = x and b = y.\\

\chapter{Functions}

\section{Types:}

ref:

\href{https://www.khanacademy.org/math/linear-algebra/matrix-transformations/inverse-transformations/v/surjective-onto-and-injective-one-to-one-functions}{serjective\&injective}
\href{URL}{text}

\begin{itemize}

	\item Reflexive funtions: $ \{ \  \forall \ \mathrm{x} \in \mathrm{A}, \  \mathrm{xRx} \ \} $
	\item Symmetric functions: $ \{ \ \forall \ \mathrm{x,y} \in \mathrm{A}, \  \mathrm{xRy} \  \& \ \mathrm{yRx} \ \} $
	\item Transitive functions: $ \{ \ \forall \ \mathrm{x,y,z} \in \mathrm{A},\ if\ \mathrm{xRy} \  \& \ \mathrm{yRz},\ then\   \mathrm{xRz}\ \} $
	\item Surjective ( "on to" ) functions:  \[ f(\mathrm{x})\rightarrow \mathrm{Y}:\{\ \forall\  \text{y $\in$ Y}\ \exists\ at\ least\ one\ \text{x $\in$X}: f(\mathrm{x})=y\  \} \] or
	      \[ im(f)=\mathrm{Y}\]
	      or
	      \[ range(f)=\mathrm{Y}\]
	\item Injective ( one-to-one)
	      \[ \{\ for\ any\   \text{y $\in$ Y}\ \exists\ at\ most\ one\ \text{x $\in$X}: f(\mathrm{x})=y\  \} \]
	      OR
	      \[ \{ y\ |\ y=f(x)=f(z),\ where\ x=z \} \]
	      
	      
\end{itemize}

\subsection{Proving if a function is a specific type}

\subsubsection{Function}

Graphically:
Vertical line tests

\href{https://www.youtube.com/watch?v=Uzlj6N5OYcM}{surjective}


\href{https://www.youtube.com/watch?v=bjATxNZp4GI}{injective}

if in f(x), assuming \( f(a) = f (b) \), then a should be equal to b.

To prove, substitute a and b in x for f(x) and equate them.

\( a=\pm b\) does not mean f(x) is injective

\section{Solving}

\href{https://www.youtube.com/watch?v=bu_6IDgXMKc}{Basic}

\section{Different functions}

\begin{tabular}{|c|c|}
	\hline
	
	\hline
	Exponential Functions & Injective \\
	Logarithmic Functions & Injective \\
	\hline    
\end{tabular}

\subsection{D \& R}
\label{sec:DRoF}

References:

\href{https://www.geeksforgeeks.org/domain-and-range-of-function}{general-DR}

\href{https://www.youtube.com/watch?v=WfgYJy035k0}{log-D}

\href{https://www.youtube.com/watch?v=rpI-X9Gn5a4}{piecewise-DR}

\begin{tabular}{| c | L | L |}
	\hline
	Types                 & \text{Domain}                                     & \text{Range}             \\
	\hline
	Polynomial Functions  & \{ x:x \in \mathbb{R} \}                          &                          \\
	Linear Functions      & ,,                                                & \{ y:y \in \mathbb{R} \} \\
	Quadratic Equations   & ,,                                                & \{ y:y=a(x-h)^2 + k,\ y
	\begin{cases} 
		y \geq k & a > 0 \\
		y \leq k & a < 0 \\
	\end{cases}\ \}                                                                                  \\
	Square root Functions & \{ y:y=\sqrt{x},\ x \geq 0\}                      & \{ y:y \geq 0 \}         \\
	Rational Functions    & \{ x:x=\dfrac{a}{b},\ a,b=\mathbb{I},\ b\neq 0 \} &                          \\
	Logarithmic Functions & \{ y:y=\log_ax,\ x > 0, x \in \mathbb{R} \}       & \{ x:x \in \mathbb{R} \} \\
	Exponential functions & ( -\infty,\infty)                                 & (0,\infty)               \\
	\hline
	
\end{tabular}

\subsection{Finding Domain}

\begin{enumerate}
	\item Step 1: First, check whether the given function can include all real numbers.
	\item Step 2: Then check whether the given function has a non-zero value in the denominator of the dfraction and a non-negative real number under the denominator of the dfraction.
	\item Step 3: In some cases, the domain of a function is subjected to certain restrictions, i.e., these restrictions are the values where the given function cannot be defined. For example, the domain of a function f(x) = 2x + 1 is the set of all real numbers (R), but the domain of the function f(x) = 1/ (2x + 1) is the set of all real numbers except -1/2.
	\item Step 4: Sometimes, the interval at which the function is defined is mentioned along with the function. For example, f (x) = 2x2 + 3, -5 < x < 5. Here, the input values of x are between -5 and 5. As a result, the domain of f(x) is (-5, 5).
\end{enumerate}

\subsection{Finding Range}

Let us consider a function y = f(x).

\begin{enumerate}

	\item Write the given function in its general representation form, i.e., y = f(x).
	\item Solve it for x and write the obtained function in the form of x = g(y).
	\item Now, the domain of the function x = g(y) will be the range of the function y = f(x).
	      
\end{enumerate}

\section{Other Types of Functions}

\begin{tabular}{|c|L|}
	\hline
	Type                  & Def                                                                                    \\
	\hline
	Exponential functions & \{ y\ |\ y=b^x,\ b is a constant, b \in ^+\mathbb{R}, b \neq 1 \}, \text{y is expoent} \\
	\hline
\end{tabular}

\subsection{Exponential Functions}

Inverse of logarithmic functions

\subsubsection{Properties}

\begin{itemize}
	\item if \(0<b<1\), y is reflected over y-axis
\end{itemize}

\begin{tabular}{|L|c|}
	\hline
	b^x=b^y \Rightarrow x=y & Equivalence Property \\
	\hline
\end{tabular}

\subsubsection{Graphing}

for:
\[f(x)=ab^{x-h}+k\]

\begin{itemize}
	\item Usual rules of a \& k apply
	\item if \(h>0\), shift h units right
	\item if \(h<0\), shift h units left
\end{itemize}

\subsubsection{Types}

Natural exponential function

Depending on the value of the base:
\begin{itemize}
	\item Exponential growth function
	\item Exponentional decay function
\end{itemize}

The line y = 0 is a hoorizontal asymptote.


\subsection{Logarithmic Functions}

Inverse of exponential functions

Synthetic division:\\

\begin{itemize}

	\item Algorithm\\
	      
	      \[
		      \begin{array}{c|rrr}
			      
			        & b & c    & d         \\
			      a &   & ab   & a(c+ab)   \\
			      \hline                   \\
			        & b & c+ab & d+a(c+ab) \\
			      
		      \end{array}
	      \]
	      
	\item Filled example:\\
	      
	      \polyhornerscheme[x=2]{2x^3+2x^2-2}
	      
\end{itemize}
\subsubsection{Properties INmcomplete}

For \(y=alog_b(x-h)+k\):

\begin{itemize}
	\item if \(0<b<1\), y is reflected over x-axis
	\item if \(a<0\), y is reflected over x-axis
	\item if \(a<0\) \& \(0<b<1\), no reflection
	\item if \(x-h=0\), vertical asymptote
\end{itemize}

Note:

\begin{itemize}
	\item \(log(x+y)\) cannot be simplified like \(log(xy)\)
	\item In logarithms, equivalences are true only for the values for which the expressions the defined.
\end{itemize}

\begin{tabular}{| l | l |}
	\hline
	\textbf{Product Rule:}                 & \(log_{a}(xy)=log_{a}(x)+log_{a}(y)\)                                                                       \\
	\hline
	\textbf{Quotient Rule:}                & \(log_{a}(\dfrac{x}{y})=log_{a}(x)-log_{a}(y)\)                                                             \\
	\hline
	\textbf{Power Rule:}                   & \(log( x) ^{n}= n log( x)\)                                                                                 \\
	\hline
	\textbf{Change of Base Formula:}       & \(\log_ab=\dfrac{log_cb}{\log_ca}\)                                                                         \\
	\hline
	\textbf{Log to Exponential Form:}      & \(\nu\upsilon \log _{a}b= c\overset {\iota \upsilon }{\operatorname* { \Leftrightarrow } }a^{c}= b\) \(to\) \\
	\hline
	\textbf{Identity Rule:}                & \(\log_{a}a=1\)                                                                                             \\
	\hline
	\textbf{Zero Rule:}                    & \(log(1)=0\)                                                                                                \\
	\hline
	\textbf{Logarithm Inverse Property:}   & \(\log_{a}a^{n}=n\)                                                                                         \\
	\hline
	\textbf{Inverse Exponent Property:}    & \({\overline{a^{\log_{a}x}=x}}\)                                                                            \\
	\hline
	\textbf{Base Switch Rule:}
	                                       & 
	\(\log_{a}b=log_ba\)
	\\
	\hline
	\textbf{Natural Log of e:}             & \(ln(e)=1\)                                                                                                 \\
	\hline
	\textbf{Reciprocal Rules:}             & \(\log_{a}\dfrac{1}{b}\cdot=-\log_{a}b\) \(\log_{1}b=-\log_{a}b\)                                           \\
	\hline
	\textbf{Power - Logarithmic Exponent:} & \(a\) \(a^{\log_{c}b}=b^{\log_{c}a}\)                                                                       \\ 
	\hline
	\textbf{Rule 14:}                      & \(\log_{xy}(a)\) )= \({\overline{\log_{a}x+\log_{a}y}}\)                                                    \\ 
	\hline
	\textbf{Rule 15:}                      & \(\log_{\dfrac{x}{y}}(a)\) 1 \(\overline{\log_{a}x-\log_{a}y}\) \(y\)                                       \\ 
	\hline
	\textbf{Logarithm of Zero:}
	                                       & 
	\(log(0)=Undefined\)
	\\ 
	\hline
\end{tabular}

\begin{tabular}{|L|}
	\hline
	log_bx=log_by \Rightarrow x=y \\
	\hline
\end{tabular}

\section{Specific functions}

\begin{tabular}{| L | L | L |}
	
	\hline
	\text{Formula} & \text{Domain} & \text{Range} \\
	\hline
	
	\dfrac{1}{ \log x}                            \\
	
	\hline
	
\end{tabular}

\section{Symmetry of Functions}

\href{https://www.statisticshowto.com/symmetry-of-a-function/}{Symmetry of Functions}

\section{Inverse Functions}

\begin{tabular}{|L|L|}
	\hline
	Exponential & Logarithmic \\
	\hline
\end{tabular}

Note: \(f^{-1} \neq \dfrac{1}{f}\)

Inverse functions can sometimes be found by reversing the order of operations done in a given function. In case of relations, interchange x and y in \((x,y)\)

For non injective functions, restrict the domain to make it injective and find its inverse.

Ref:

\href{https://www.youtube.com/watch?v=ukEtad_aml4}{Graphing Inverse Functions}

\url{https://www.youtube.com/watch?v=GsIo3B46yjU}

\url{https://www.khanacademy.org/math/algebra-home/alg-functions/alg-finding-inverse-functions/a/finding-inverse-functions}

\section{End behaviour}
The end behavior of a function describes the behavior of the graph of the function at the "ends" of the x-axis. 

ref:
https://www.khanacademy.org/math/algebra2/x2ec2f6f830c9fb89:poly-graphs/x2ec2f6f830c9fb89:poly-end-behavior/a/end-behavior-of-polynomials

\section{Zeroes \& Multiplicity}

ref:
https://www.khanacademy.org/math/algebra2/x2ec2f6f830c9fb89:poly-graphs/x2ec2f6f830c9fb89:poly-intervals/a/zeros-of-polynomials-and-their-graphs

\section{Graphing}
Using end behaviour:

\begin{tabular}{| c | L | L |}
	\hline
	
	Power & +                  & -                    \\
	\hline
	
	Odd   & \downarrow\uparrow & \uparrow\downarrow   \\
	Even  & \uparrow\uparrow   & \downarrow\downarrow \\
	
	\hline
\end{tabular}

ref:

\url{https://www.youtube.com/watch?v=a5x4lwnvHM0}

Exponential and Logarithmic functions

\url{https://www.youtube.com/watch?v=ymXD6xCmzJE}

\subsection{Transformations}

A solution to a system of linear equaations is an ordered pair that is a solution to each individual equations in it.

ref:

\url{https://www.youtube.com/watch?v=Tmdrjs9xufc&t=849s}

\section{System of equations}

\subsection{Solving by the Graphical Method}

When 2 lines are drawn in a rectangular coordinate system, three geometric relations are possible;
\begin{itemize}
	\item They intersect at exactly 1 point
	\item They are parallel
	\item They may coincide, i.e., intersect at infinitely many points.
\end{itemize}

\href{https://www.youtube.com/watch?v=bq5gDsEdN3Q&t=100s}{No of solutions}

Systems of linear equaations are:
\begin{itemize}
	\item Consistent, if having 1 or more solutions
	\item Inconsistent, if having no solution.
	\item Dependent, if they represent the same line
	\item Independent, if they represent different lines
\end{itemize}

ref:

\href{https://www.cuemath.com/ncert-solutions/i-for-which-values-of-a-and-b-will-the-following-pair-of-linear-equations-have-an-infinite-number-of-solutions-2x-3y-7-a-b-x-a-b-y-3a-b-2-ii-for-which-value-of-k-will-the/}{Solve for to get type}

\href{https://brainly.com/question/46551990}{Solve for to get type}

\subsection{Solving by the The Substitution Method}

After substitution, if you get an identity:
\begin{itemize}
	\item x=constant, system has 1 solution
	\item x=x, system has infinite solution
	\item \(constant_1=constant_2\), system has no solution
\end{itemize}

\subsection{Solving by the Addition Method}

After addition, if you get an identity:
\begin{itemize}
	\item x=constant, system has 1 solution
	\item x=x, system has infinite solution
	\item \(constant_1=constant_2\), system has no solution
\end{itemize}


\chapter{Equations}

Zero product rule:
\[If\ ab=0,\ then\ a=0\ or\ b=0\]

\chapter{Trigonometry}

\begin{flalign*}
	 & \theta=\dfrac{l}{r}                                                                              \\
	 & x\degree=[x\times60]\text{'}=[x\times60\times60]\text{'}\text{'}                                 \\
	 & \text{Radian measure}=\dfrac{\pi}{180}\times\text{Degree measure}                                \\
	 & \text{Degree measure}=\dfrac{180}{\pi}\times2                                                    \\
	 & \csc x=\dfrac{1}{\sin x}, x\neq n\pi, \text{where }n \text{ is any integer.}                     \\
	 & \sec x=\dfrac{1}{\cos x}, x\neq(2n+1) \dfrac{\pi}{2} , \text{where }n \text{ is any integer.}    \\
	 & \tan x=\dfrac{\sin x}{\cos x}, x\neq(2n+1)\dfrac{\pi}{2}, \text{where }n \text{ is any integer.} \\
	 & \cot x=\dfrac{\cos x}{\sin x}, x\neq n \pi, \text{where } n \text{ is any integer.}              \\
\end{flalign*}

\section{Sum and Difference Formulas}

Pythagorean Formulas:
\begin{flalign*}
	 & \sin^2x+\cos^2x=1 \\\
	 & 1+\tan^2x=\sec^2x \\
	 & 1+\cot^2x=\csc^2x \\
\end{flalign*}

Reduction Formula:

Given \(A\sin x+B\cos x\), we have for \(k \in \prescript{+}{}{\mathbb{R}}\):
\begin{flalign*}
	 & A\sin x+B\cos x=k\sin (x+\alpha)                                                                                 \\
	 & where\ k=\sqrt{A^2+B^2},\ and\ \alpha\ \text{satisfies } \cos \alpha=\frac{A}{k},\ and\ \sin \alpha =\frac{B}{k} \\
\end{flalign*}

Addition and Subtraction from $\pi$:\\

\includegraphics[width=1\linewidth]{adfpi0.png}

\begin{tabular}{|L|L|}
	\hline    
	Initial          & Final       \\
	\hline
	\sin(\theta-\pi) & -\sin\theta \\
	\cos(\theta-\pi) & -\sin\theta \\
	\tan(\theta-\pi) & -\sin\theta \\
	\csc(\theta-\pi) & -\sin\theta \\
	\sec(\theta-\pi) & -\sin\theta \\
	\cot(\theta-\pi) & -\sin\theta \\
	\hline    
\end{tabular}

\begin{flalign*}
	 & \cos (\dfrac{\pi}{2}+x) =-\sin x \\
	 & \sin (\dfrac{\pi}{2}+x) =\cos x  \\
	 & \cos (\pi-x) =-\cos x            \\
	 & \sin (\pi-x) =\sin x             \\
\end{flalign*}
Rather than studying these, study this:\\
For $n\pi$, where \(n\in\textbf{N}\Rightarrow\) unchanged\\
For $n\dfrac{\pi}{2}$, where \(n\in\textbf{Odd N}\Rightarrow\) changed\\
When changed:
\begin{flalign*}
	 & \sin\leftrightarrow\cos \\
	 & \sec\leftrightarrow\csc \\
	 & \tan\leftrightarrow\cot \\
\end{flalign*}

\section{Angles, Trig. \& Quadrants}

ref:

\href{https://www.youtube.com/watch?v=1MoCe5QvHU4}{Determine your quadrant when given constaints on trig. functions}

\href{https://www.youtube.com/watch?v=FTk2fjvCkQM}{Determine the Quadrant in which an Angle Lies}

\href{https://www.youtube.com/watch?v=1MoCe5QvHU4}{Simple explanation}

\begin{figure}
	\includegraphics[width=0.2\linewidth]{image.png}
	\caption{ASTC = "Add Sugar To Coffee" = All, Sin, Tan, Cot}
	\label{fig:enter-label}
\end{figure}


\href{https://www.youtube.com/watch?v=qW6Ua50fTw8}{How to} memorize the table below:

\begin{table}[!ht]
	\centering
	\begin{tabular}{|c|c|c|c|c|c|c|c|}
		\hline
		Angles (in Degrees) & 0°          & 30°        & 45°        & 60°        & 90°         & 180°        & 270°        \\ \hline
		Angles (in Radians) & 0           & $\pi$/6    & $\pi$/4    & $\pi$/3    & $\pi$/2     & $\pi$       & 3$\pi$/2    \\ \hline
		Sin                 & 0           & 1/2        & 1/$\sqrt2$ & $\sqrt3$/2 & 1           & 0           & -1          \\ \hline
		Cos                 & 1           & $\sqrt3$/2 & 1/$\sqrt3$ & 1/2        & 0           & -1          & 0           \\ \hline
		Tan                 & 0           & 1/$\sqrt3$ & 1          & $\sqrt3$   & Not Defined & 0           & Not Defined \\ \hline
		Cot                 & Not Defined & $\sqrt3$   & 1          & 1/$\sqrt3$ & 0           & Not Defined & 0           \\ \hline
		Cosec               & Not Defined & 2          & $\sqrt2$   & 2/$\sqrt3$ & 1           & Not Defined & -1          \\ \hline
		Sec                 & 1           & 2/$\sqrt3$ & $\sqrt2$   & 2          & Not Defined & -1          & Not Defined \\ \hline
	\end{tabular}
\end{table}

\section{Domain \& Range}

\includegraphics[width=1\linewidth]{trigdr.png}

\includegraphics[width=1\linewidth]{inverse trig domain and range.png}

\href{https://en.wikipedia.org/wiki/Trigonometric_functions}{source}

\href{https://www.youtube.com/watch?v=Vw-RwPBWS8g&t=13s}{DR of sin and cos}

\section{\href{https://www.youtube.com/watch?v=fo_q9mEAFp4&t=2472s}{Graphing}}

\section{Specific Problems}

\href{https://www.youtube.com/watch?v=igkTyHWXx7Y}{How to} solve trig equations with a calculator (and the unit circle)


\subsection{Simplify Trigonometric Expressions}

Methods include:
\begin{itemize}
	\item Using Quotient and Reciprocal identitites
	\item Adding fractional expressions
	\item Factoring
\end{itemize}

\subsection{Verify Trigonometric identities}

Methods include:
\begin{itemize}
	\item Simplify negative arguments
	\item Combining factors
	\item Using conjugates
	\item Logarithmic manipulation, by treating them as normal variables, for those involving them
	\item By working each side independently
\end{itemize}

\subsection{Write an algebric expression as a trigonometric expression}

\section{Genereal structure}

\[y=A<trig>[B(x-C)]+D\]

Here:

\begin{tabular}{|c|L|}
	\hline
	Time period, \(t\)      & \dfrac{2\pi}{B} \\
	Horizontal shift, Phase & C               \\
	Vertical shift          & D               \\
	Amplitude               & A               \\
	
	\hline
\end{tabular}

Periodicity is a property of a function. The period (time period) is a specific number associated with that property.

\chapter{Inequalities}

\section{Solving Linear Inequalities}

\href{https://www.youtube.com/watch?v=DrZJKdXlZ3I}{Basic}

\chapter{Straight Lines}

Midpoint of a line with two points $(x_1,y_1),(x_2,y_2)$ :
\[\{\dfrac{x_1+x_2}{2},\dfrac{y_1+y_2}{2}\}\]

\chapter{Limits}
\begin{flalign*}
	 & \operatorname*{lim}_{x\to a}{\dfrac{x^{n}-a^{n}}{x-a}}=n a^{n-1} \\
	 & \operatorname*{lim}_{x\to0}{\dfrac{\sin x}{x}}=1                 \\
	 & \operatorname*{lim}_{x\to0}{\dfrac{1-\cos x}{x}}=0               \\
	 & \operatorname*{lim}_{x\to0}{\dfrac{\tan x}{x}}=1                 \\
\end{flalign*}
\chapter{Derivatives}
First principle of Derivatives:
\begin{flalign*}
	 & f^{\prime}(x)={\dfrac{df(x)}{d x}}=\operatorname*{lim}_{h\to0}{\dfrac{f(x+h)-f(x)}{h}} \\
\end{flalign*}
Some general formulas:
\begin{flalign*}
	 & {\dfrac{d}{d x}}(x^{n}){=}n x^{n-1}           \\
	 & \dfrac{dx}{dx}=1                              \\
	 & \dfrac{dc}{dx}=1\text{( where c is a const )} \\
\end{flalign*}
Some trig formulas:
\begin{flalign*}
	 & \dfrac{d}{dx}\sin x=\cos x               \\
	 & \dfrac{d}{dx}(\cos x)=-\sin x            \\
	 & \dfrac{d}{dx}(\tan x)=\sec^2 x           \\
	 & \dfrac{d}{dx}(\cot x)=-\csc^2 x          \\
	 & \dfrac{d}{dx}(\sec x)=(\sec x)(\tan x)   \\
	 & \dfrac{d}{dx}(\csc x)=(-\csc x )(\cot x) \\
\end{flalign*}
Mnemonics:
\begin{itemize}
	\item The co in the above trig’s is a ‘cori’ ( Malayalam for chicken ) which is a negative while studying
	\item ( sec sec tan, cosec cosec cot )
\end{itemize}
Pick one, equate the rest

\chapter{Statistics}
Deviation around a specific value 'a', \( x-a \) 
mean x by step-deviation method:
\begin{flalign*}
	\overline{x}=a+\dfrac{1}{\mathrm{N}}\sum_{i=1}^{n}f_id_i\times{h}
\end{flalign*}
Standard deviation of a discrete frequency distribution\\
\begin{flalign*}
	\sigma = \sqrt{\dfrac{1}{\mathrm{N}}\sum_{i=1}^nf_i(x_i - \overline{x})^2} \text{ ,where N} =\sum_{i=1}^{n}f_{i} 
\end{flalign*}
\chapter{Matrices}
DEFINITION:\\
A matrix is an ordered rectangular array of numbers or functions. The
numbers or functions are called the elements or the entries of the matrix.\\
DEFINITION: \\
If \(A = [a_{ij}]\) be an \(m\times n\) matrix, then the matrix obtained by interchanging the rows and columns of A is called the transpose of A. Transpose of the matrix A is denoted by A' or (\(A^T\)). In other words, if \(A = [a_{ij}]_{m\times n}\), then \(A^\prime=[a_{ji}]_{n\times m}\)\\
DEFINITION:\\
A square matrix \( \mathrm{A}=[a_{ij}]\) is said to be symmetric if A' = A, that is, \([a_{ij}]=[a_{ji}]\) for all possible values of i and j.\\
\section{Types of Matrices}
Note: term 'main diagonal'\\
Column matrix \( \Rightarrow \mathrm{A}=[a_{ij}]_{m\times1}\)\\
Row matrix\( \Rightarrow \mathrm{A}=[a_{ij}]_{1\times n} \)\\
Square matrix\( \Rightarrow  \mathrm{A}=[a_{ij}]_{m\times m} \) , is a square matrix of order m\\
Diagonal matrix\( \Rightarrow \mathrm{A}=[a_{ij}]_{n\times n}\text{, }b_{ij}=0 \), where \(i\neq j\)\\
Scalar matrix\( \Rightarrow  \mathrm{A}=[a_{ij}]_{n\times n} \), if \\
\begin{flalign*}
	 & b_{ij}=0  \quad \text{when }i\neq j \\
	 & b_{ij}=k \quad \text{when }i=j      \\
\end{flalign*}\\
Identity matrix\( \Rightarrow\mathrm{A}=[a_{ij}]_{n\times n}\text{, } a_{ij}=\begin{cases}1&\mathrm{if}\quad i=j\\0&\mathrm{if}\quad i\neq j\end{cases} \)\\
Zero matrix (O)\\
\section{Properties of Matrices}
Multiplication of matrices is not commutative, i.e.,\\

\begin{equation*}
	AB\neq BA    
\end{equation*}
\begin{equation*}
	\begin{split}
		&(\mathrm{A}^{\prime})^{\prime}=\mathrm{A}\\
		&(k\mathrm{A})^{\prime}=k\mathrm{A}^{\prime} (\mathrm{where} k\mathrm{~is~any~constant})\\
	\end{split}
	\quad \quad
	\begin{split}
		&(\mathrm{A}+\mathrm{B})^{\prime}=\mathrm{A}^{\prime}+\mathrm{B}^{\prime}\\
		&(\mathrm{A}\mathrm{B})^{\prime}=\mathrm{B}^{\prime} \mathrm{A}^{\prime}\\
	\end{split}
\end{equation*}
\begin{align*}
	 & \mathrm{A}+\mathrm{A}^\mathrm{T}=\mathrm{B}\text{, where \(\mathrm{B}\) is a symmetric matrix}                                                                                                     \\
	 & \mathrm{A}-\mathrm{A}^\mathrm{T}=\mathrm{C}\text{, where \(\mathrm{C}\) is a skew symmetric matrix}                                                                                                \\
	 & \dfrac{1}{2}(\mathrm{A}+\mathrm{A}^\mathrm{T})+\dfrac{1}{2}(\mathrm{A}-\mathrm{A}^\mathrm{T})=\mathrm{A}\text{, where \(\mathrm{A}\) is the sum of a symmetric matrix and a skew symmetric matrix}
\end{align*}

\section{Operations with Matrices}

\subsection{Gaussian Elimination \& Row echolon form}

\url{https://www.youtube.com/watch?v=eDb6iugi6Uk&list=PL0o_zxa4K1BU5sTWZ2YxFhpXwsnMfMke7&index=208}

\subsection{Gaussian Elimination \& Row Echelon Form}

\url{https://www.youtube.com/watch?v=eYSASx8_nyg&list=PL0o_zxa4K1BU5sTWZ2YxFhpXwsnMfMke7&index=210}

\part{Calculus}
\chapter{Continuity}
\begin{flalign*}
	 & LHL=RHL=f(a)                        \\
	 & \Rightarrow\text{f is cont. at x=a}
\end{flalign*}
Polynomial functions is continuous\\
Rational Functions is continuous\\
"Sine” Function is continuous\\
“Cosine” Function is continuous\\
“Tangent” Function is continuous over it's domain\\
If f(x), g(x) is cont, then fog(x), gof(x) is cont.\\

\part{Combinatorics}

Fundamental Principle of Counting:

If 1 event can occur in m different ways and another in n different ways, then the number of ways that the 2 events can occur in sequence is \(m \cdot n\)

\chapter{Permutations}

A permutation is an arrangement of objects in a definite order.

The number of permutations of \(n\) distinct elements is \(n!\).

Formula:

\[\prescript{n}{}P_r=\dfrac{n!}{(n-r)!}\ or\ \prescript{n}{}P_r=\overbrace{n(n-1)(n-2)...(n-r+1)}^{r\text{ factors}} \]

NOTE: The above works on the assumption that no item is selected more than once and that each are distinguishable.

\chapter{Combinations}

A combination is an arrangement of objects in a indefinite order.

Number of combinations of \(n\) elements taken \(r\) at a time:

\[ \prescript{n}{}{C}_r = \dfrac{n!}{r! \cdot (n-r)!}\ or\ \prescript{n}{}{C}_r = \dfrac{\prescript{n}{}{P}_r}{r!} \]

\part{Probability}

\chapter{Definitions}

\begin{itemize}
	\item Probability: The value assigned to an event that quantifies how likely it is to occur.
	\item Experiment: a test for an outcome where the results are uncertain.
	\item Sample space: of an experiment
	\item Theoretical probability of event E:
	\item Let \(S\) be sample space and \(E\) be a subset of \(S\). Then:
	      \[p(E)=\dfrac{n(E)}{n(S)}\]
	\item Complimentary events
	\item Let \(E\) be an event of \(S\). \(\overline{E}=S-E\) ( E compliment, also represented \(\sim E\ or\ E'\)). It follows that:
	      \[
		      P(E)+p(\overline{E})=1
	      \]
	      
\end{itemize}
\begin{tabular}{|c|c|}
	\hline
	Classical Probability                     & 
	Emperical Probability                               \\
	\hline
	\(E \& S\) from theory and mathematics    & 
	\(E \& S\) from experience and observation          \\
	Describes how likely an event is to occur & 
	Describes how frequently an event actually occurred \\
	Guess                                     & 
	Observation                                         \\
	\hline
\end{tabular}

\chapter{Some operations}

\begin{itemize}
	\item Probability of \((A\ or\ B)\)
	      Given events from the same \(S\):
	      \[
		      P(A \cup B)=p(A)+P(B)-P(A \cap B)
	      \]
	      Note: If A and B are mutually exclusive, then \(P(A \cap B)=0\)
	\item Probability of a sequence of sequential independent events:
	      \[P(A\ and\ B)=P(A)\cdot P(B)\]
\end{itemize}

\end{document}
